\documentclass[a4paper,12pt]{article}

\usepackage[utf8]{inputenc}
\usepackage[IL2]{fontenc}
\usepackage[czech]{babel}
\usepackage[margin=2.5cm,top=2cm,a4paper]{geometry}
\usepackage{times}
\usepackage{graphicx}
\usepackage{tikz}
\usepackage{enumitem}
\usepackage{ulem}

\pagenumbering{gobble}

\renewcommand{\ULthickness}{1.35pt}
\setlist{noitemsep,topsep=3pt}


\usetikzlibrary{calc}
\newcommand{\checkbox}[2][5]{
  \hfill\tikz[baseline,anchor=base]{
 
\foreach \i in {1,...,#1} {
  \node[rectangle,draw, minimum width=4mm,minimum height=4mm] (\i) at ($(\i*7mm,1mm)$) {\textcolor{white}{}};
  }
  \foreach \i in {#2} {
  \node[cross] at (\i) {};
}
}\hspace{50pt}~%
}

\begin{document}%
\parindent=0pt
\tikzset{cross/.style={path picture={ 
  \draw[black]
(path picture bounding box.south east) -- (path picture bounding box.north west) (path picture bounding box.south west) -- (path picture bounding box.north east);
}}}
\tikz[remember picture,overlay] {\node[anchor= north east] at ($(current page.north east)+(-3.5,-2)$) {\includegraphics[width=3cm]{vse}};}%
{\Large \sc \textbf{oponentský posudek bakalářské práce}}
\vskip20pt
\textbf{Katedra:} \quad \textbf{KEKO}
\vskip7pt
\textbf{Student:} \quad \textbf{ }
\vskip7pt
\textbf{Oponent bakalářské práce:} \quad \textbf{ }
\vskip3pt
\textbf{Pracoviště a funkce:} \quad \textbf{ }
\vskip7pt
\textbf{Téma bakalářské práce:} \quad \textbf{ }
\vskip30pt

\hspace*{10.9cm}%
\tikz[anchor=base,baseline,remember picture]{
\foreach \i in {1,...,5} {
  \node[] (h\i) at ($\i*(7mm,0)$) {\textbf{\i}};
  }
} \\
\uline{\textbf{Hodnocení práce}}
\hspace*{6.6cm} \mbox{\footnotesize  hodnocení: 1 = nejlepší, 5 = nejhorší} \\
\vspace*{-0.25cm}
\begin{enumerate}
  \item Formulace cílů práce a metodika zpracování \checkbox{1,2} \\
\vspace*{-0.25cm}
  \item Práce s daty a informacemi \checkbox{} \\
\vspace*{-0.25cm}
  \item Členění práce (kapitoly, podkapitoly, odstavce) \checkbox{} \\
\vspace*{-0.25cm}
  \item Práce s odbornou literaturou (citace, norma) \checkbox{} \\
\vspace*{-0.25cm}
  \item Úroveň jazykového zpracování \checkbox{} \\
\vspace*{-0.25cm}
  \item Přesnost formulací a práce s odborným jazykem \checkbox{} \\
\vspace*{-0.25cm}
  \item Formální zpracování \checkbox{} \\
\vspace*{-0.25cm}
  \item Splnění cílů práce, závěry práce a jejich formulace \checkbox{} \\
\vspace*{-0.25cm}
  \item Souhrn a klíčová slova odpovídají obsahu práce \checkbox{} \\
\vspace*{-0.25cm}
  \item Celkové hodnocení práce známkou (1, 2, 3, 4) \checkbox[4]{1}\hspace*{7mm}
\end{enumerate}
\vskip15pt
\uline{\textbf{Otázky k obhajobě}}
%\begin{enumerate}[label=\roman*)]
%\end{enumerate}
\vskip36pt

\uline{\textbf{Další připomínky, vyjádření, náměty k obhajobě práce}}
\vskip45pt
%\parindent=1em

\textbf{Datum:}
\vskip10pt
\hfill \textbf{Podpis oponenta bakalářské práce}

    
\end{document}

